\section{Conclusion}

This study demonstrated that with structured mentorship, hands-on practice, and
exposure to real-world workflows, it is possible to effectively prepare
newcomers for meaningful contributions to complex Free Software projects like
the Linux kernel. The approach adopted - a university course combining tutorials
by experienced developers, in-loco workshops, and personalized guidance -
successfully demystifies FLOSS development, fostering hard and soft skill growth
and building contributor confidence. While the results are promising, further
validation is needed to assess scalability beyond academic environments.
Nonetheless, this work offers a replicable model to sustainably onboard
long-lasting contributors to Free Software ecosystems.
