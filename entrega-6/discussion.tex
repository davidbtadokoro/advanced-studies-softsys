\section{Discussion}

After describing the results of this research work, we present our analysis of
them, as well as their implications and limitations.

\subsection{Key Findings}

First of all, having a group of students that, in its majority, had little to no
experience in Free Software development - even though most were familiar with
the concept - allows us to consider the experience on the course as central (at
least impactful) to the results collected at the end of the course.

RQ1 asked about the effective teaching techniques and resources to introduce
newcomers to Free Software projects. The results answer RQ1 by showing that most
students agreed that close mentorship throughout the course, along with in-loco
workshops with tutorials devised by veterans, were essential in their
development as Free Software contributors. Some individual reports even showed
that, beyond the support of teaching assistants, the help of peers (i.e., fellow
students of the course) by pairing or working in groups was effective in
learning and contributing, enforcing, even more, the importance of the in-loco
workshops.

Results show that starting the experience with contributions to the Linux kernel
project was daunting to students. Technical difficulties inherent in the
workflows of the Linux kernel, especially the environment setup part, and vague
or even misleading tutorial instructions were major hurdles. However, student
reports (corroborated by the observations of the teaching assistants and the
professor) show that this rough start elevated the students' hard and soft
skills, making them continuously more independent as the course progressed.

Beyond that, presenting the diverse Free Software development models from the
Linux kernel, an email-based model with higher technical requirements, to
Web-based models, which are considered friendlier to new contributors, gave a
comprehensive and complete view of the Free Software ecosystem. The course
experience made students considerably more confident in furthering their
contributions to Free Software projects.

Regarding RQ2, which covered the hard and soft skills necessary for successfully
contributing to Free Software projects, the results highlight the importance of
the git version control system. The ability to manage a Linux system and use CLI
tools, like the ones from the GNU coreutils, is essential for executing tasks
related to the workflows of Free Software projects and troubleshooting problems.
Students also pointed out hard skills like C programming language or device
driver development that are specific to the Linux project, but we argue that
every software project, not only Free Software ones, has intricate specifics
that compose hard skills requirements. From students' reports, soft skills like
good communication are paramount for success in Free Software ecosystems.

\subsection{Implications}

Introducing a group of developers to Free Software projects and communities is a
difficult task that demands preparation, dedication, and coordination. Beyond
introducing newcomers, equipping them with the necessary hard and soft skills
for impactful contributions to these projects requires a more immersive
approach. 

Through this study, we presented a successful approach to training long-lasting
contributors to Free Software projects, especially (but not limited to) the
Linux kernel project.

This approach can be consolidated as a four-month course composed of:

\begin{enumerate}
    \item In-loco workshops with tutorials devised by veteran developers of Free
    Software projects;
    \item Hands-on experience with Free Software projects by sending
    contributions and participating in the review process;
    \item Exposition of projects with different models of development for a
    comprehensive view of the complete Free Software ecosystem;
    \item Monitoring of the progress in activities;
    \item Close mentoring with teaching assistants.
\end{enumerate}

Previous knowledge and experience, although desirable, are not requisites for
this approach, as hard and soft skills are naturally developed through the
course.

\subsection{Limitations}

This study relies on aspects that can limit the triumphant applicability of the
presented approach.

The proposed approach was implemented in a university course that included
undergraduate and graduate students. Due to this, the objects of study, i.e.,
the students, had an influenced motivation in terms of engaging in the course
experience. On the one hand, students could have opted to use the minimal effort
required to complete the course, which negatively impacts the results. On the
other hand, students aspiring for good grades could have extra motivation and
engagement in the course activities, which would positively impact the results.
In any case, we argue that motivation and dedication are crucial for developers
aiming to enter any Free Software project.

Another limitation to consider is the number of students who completed the
course, which was 24. With more developers, the approach would certainly need
more teaching assistants, but it is impossible to affirm to what extent the
approach can escalate. With fewer developers, say, five, the approach could be
overkill or even introduce overheads in the experience, as the teaching
techniques described were envisioned for a group of considerable size.
