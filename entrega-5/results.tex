\section{Results}
\label{sec:results}

The student survey and the teaching assistants' and professors' observations
provide complementary insights into the students' development throughout the
course. Beyond that, the students' blog posts offer fine-grained observations
from their perspectives, which contribute to triangulating results that emerged
from the other two data sources.

Sections \ref{sec:results:role-mentors-workshops} to
\ref{sec:results:prior-knowledge-perceptions} show the results from the survey
responses merged with the teaching assistants' and the professor's observations.
Section~\ref{sec:results:blog-posts} brings results that cluster insights
detected in students' blog posts.

\subsection{Prior knowledge and shifting perceptions on Free Software}
\label{sec:results:prior-knowledge-perceptions}

A significant majority of respondents (70\%) indicated they were familiar with
the concept of free software before the course. However, only 25\% reported
having ever contributed to a Free Software project before.

At the end of the course, despite this limited prior experience, 75\% of the
students agreed that they would be willing to use the experience gained to
contribute to Free Software projects. Similarly, 75\% felt comfortable
contributing to Free Software by the end of the course. These results are
consistent with the teaching assistants' observations that students began the
course with a somewhat mystified and distant perception of Free Software,
viewing it as complex and inaccessible. Over time, however, they came to
understand Free Software development as approachable and realistically within
their reach, particularly after gaining hands-on experience.

\subsection{Linux experience and confidence to contribute}

At the start of the course, 70\% of students had at least one year of experience
using Linux, but only 25\% had some experience with Linux from a development
standpoint. By the end of the course, 65\% agreed that the course had prepared
them to contribute to a Linux-related project (a subsystem or sub-project of
Linux). Nonetheless, only 40\% expressed intent to do so voluntarily or
professionally.

Teaching assistants noted that students initially struggled with fundamental
command-line commands such as utilities from the \textit{GNU coreutils}, like
\texttt{cd}, \texttt{ls}, and \texttt{echo}, as well as overall familiarity with
a \textit{Command-Line Interface} (CLI) approach to do software development.
They also found it challenging to troubleshoot issues arising during tutorial
workshops, such as package discrepancies across distributions. From the
perspective of the teaching assistants, these difficulties were, at large,
surpassed as the course progressed, and students grew noticeably more
independent and confident over time.

\subsection{The importance of \texttt{git}}

A core tool used throughout the whole course was \texttt{git}, and teaching
assistants pointed out that the majority of students were unfamiliar with it,
both regarding \textit{Version Control System} (VCS) concepts as well as
practical use of the tool and its commands.

Regarding the respondents' self-assessed skills, 85\% reported having some
(50\%) or extensive (35\%) experience using \textit{git} locally prior to the
course. Additionally, 70\% had some (55\%) or significant (15\%) experience with
web-based code hosting platforms like GitHub and GitLab; however, only 55\% had
some (35\%) or substantial (20\%) familiarity with pull request workflows on
these platforms. Conversely, 80\% had no prior experience with email-based
contribution workflows, the primary method used in the Linux kernel project.
Among 20 respondents, 90\% agreed that the course experience improved their
\textit{git} skills, and the rest (10\%) did not respond. By the course's end,
85\% of respondents affirmed that \textit{git} skills were among the most
important for success in the course.

\subsection{Role of mentorship and in-loco workshops}
\label{sec:results:role-mentors-workshops}

Mentorship and in-person workshops emerged as two of the course's most impactful
elements. 90\% of respondents agreed that mentorship helped complete
assignments, and 85\% considered it critical. Similarly, 85\% and 80\% of the
respondents signaled that the class teaching assistants' support and workshops,
respectively, were the biggest strengths of the course.

The teaching assistants confirmed that while the early stages of the course
required intensive support, students gradually developed autonomy, with some
independently finding and contributing to a Free Software project in the final
phase with minimal guidance.

\subsection{Student self-reported experiences in blog posts}
\label{sec:results:blog-posts}

The students' blog posts provide a multifaceted insight into the complexity and
educational value of contributing to the Linux kernel, related projects, and
Free Software in general. This section explores the main themes emerging from
their reflections: initial challenges, learning outcomes, collaboration
dynamics, and motivation for future contributions.

\subsubsection{Navigating initial complexity and technical hurdles}

Many students reported a steep learning curve associated with the initial setup
and contributions to the Linux kernel. The compilation and deployment of a
custom-built kernel and modules were frequently cited as significant technical
challenges:

\begin{itemize}
    \item Tooling and environment setup: Several students struggled to configure
    virtual machines, manage disk space, or resolve dependency issues during
    kernel compilation. Problems with outdated branches and unclear tutorial
    instructions also caused setbacks, as described by \textit{student 2} and
    \textit{student 4};
    \item Debugging and troubleshooting: Errors such as missing macros or
    package build failures were common. Students emphasized that many problems
    stemmed from gaps in tutorial coverage or their inexperience, as
    \textit{student 12} reflected on the need to learn about language and file
    management intricacies required for kernel compilation;
    \item Patience and persistence: The troubleshooting process demanded
    patience, with students often needing to restart tutorials, switch methods,
    or pair-program to overcome obstacles. The collective effort and peer
    support were crucial in overcoming frustrating moments, as highlighted by
    \textit{student 1} and \textit{student 9}.
\end{itemize}

\subsubsection{Growth in skills and familiarity with Free Software workflows}

Despite the aforementioned difficulties, the contributions significantly boosted
the students' technical proficiency and understanding of open-source development
workflows:

\begin{itemize}
    \item Deepening knowledge of Linux and CLI: Multiple students gained
    valuable hands-on experience with bash scripting, \texttt{git} intermediate
    and advanced skills, including rebasing and patch management (fundamental to
    many Free Software workflows), and kernel internals. The exposure to
    managing large codebases and coordinating with maintainers provided a
    concrete understanding of real-world software engineering practices as
    reported by \texttt{student 1}, \textit{student 5}, and \textit{student 6)};
    \item Understanding Free Software workflows: The students learned not only
    the technical aspects but also the social and procedural dimensions of
    contributing to Free Software: formatting contributions according to code
    style rules, submitting code for review, incorporating feedback from
    maintainers, and appreciating the importance of documentation and community
    engagement as brought up by \textit{student 3}, \textit{student 6}, and
    \textit{student 7};
    \item Confidence and motivation: The successful submission of patches, even
    with the required iterations and suggested changes, generated a sense of
    achievement and confidence. Students felt more empowered to explore other
    Free Software projects and appreciated the accessibility of the ecosystem
    more than initially expected, as mentioned by \textit{student 5},
    \textit{student 6}, and \textit{student 15}.
\end{itemize}

\subsubsection{Collaboration and peer support as key success factors}

The reflections underscore the importance of collaborative learning and support
networks:

\begin{itemize}
    \item Learning with peers: In cases where individual progress stalled due to
    technical difficulties or setup issues, pairing with classmates or relying
    on group work proved to be effective strategies to continue learning and
    contributing, as reported by \textit{student 4} and \textit{student 9};
    \item Support from teaching assistants and the professor: Guidance from
    professors and teaching assistants was highlighted as essential to
    navigating tricky points and clarifying misunderstandings, especially
    related to updating branches or interpreting error messages, as mentioned by
    \textit{student 2}, \textit{student 6}, and \textit{student 11};
    \item Free Software communities and ecosystems dynamics: Some students
    expanded their perspectives to consider broader aspects of Free Software
    projects - such as maintainers' responsiveness and project sustainability -
    recognizing the value of incremental contributions beyond feature
    development, as highlighted by \textit{student 3}.
\end{itemize}

\subsubsection{Tutorials improvements}

Some blog posts point toward the need for improving the tutorials used as
educational resources:

\begin{itemize}
    \item More comprehensive tutorials: Students noted that while tutorials
    effectively outlined objectives and commands, they often lacked in-depth
    troubleshooting guidance and anticipation of common pitfalls. This led to
    confusion or repeated errors that might be avoidable with better
    documentation, as brought up by \textit{student 2}, \textit{student 8}, and
    \textit{student 13}.
\end{itemize}
